%%=============================================================================
%% LaTeX sjabloon voor bachelorproef, HoGent Bedrijf en Organisatie
%% Opleiding Toegepaste Informatica
%%=============================================================================

\documentclass[fleqn,a4paper,12pt]{book}

\usepackage{listings}
\usepackage{xcolor}

\input{structure}

\newcommand{\student}{Bert Vervaele}

\newcommand{\promotor}{Bert Van Vreckem}

\newcommand{\copromotor}{Martin Van Den Abeele}

\newcommand{\instelling}{oXya}

\newcommand{\titel}{Proactief monitoren aan de hand van log files en Elastic stack}

\newcommand{\datum}{2 juni 2017}

\newcommand{\academiejaar}{2016-2017}

\newcommand{\examenperiode}{2}

%%=============================================================================
%% Inhoud document
%%=============================================================================

\begin{document}

%---------- Titelblad ---------------------------------------------------------
\inserttitlepage

%---------- Samenvatting, voorwoord -------------------------------------------
\usechapterimagefalse
%%=============================================================================
%% Samenvatting
%%=============================================================================

%% TODO: De "abstract" of samenvatting is een kernachtige (~ 1 blz. voor een
%% thesis) synthese van het document.
%%
%% Deze aspecten moeten zeker aan bod komen:
%% - Context: waarom is dit werk belangrijk?
%% - Nood: waarom moest dit onderzocht worden?
%% - Taak: wat heb je precies gedaan?
%% - Object: wat staat in dit document geschreven?
%% - Resultaat: wat was het resultaat?
%% - Conclusie: wat is/zijn de belangrijkste conclusie(s)?
%% - Perspectief: blijven er nog vragen open die in de toekomst nog kunnen
%%    onderzocht worden? Wat is een mogelijk vervolg voor jouw onderzoek?
%%
%% LET OP! Een samenvatting is GEEN voorwoord!

%%---------- Samenvatting -----------------------------------------------------
%%
%% De samenvatting in de hoofdtaal van het document

\chapter*{\IfLanguageName{dutch}{Samenvatting}{Abstract}}

De analyse van logfiles is een onderwerp dat recent aan belang gewonnen heeft en oXya,de opdrachtgever achter de bachelorproef,had dit dan ook graag in meer detail bekeken. Hier zijn enkele tools voor te vinden maar de Elastic stack trok hun aandacht. Dit omwille van de schaalbaarheid met grote hoeveelheden data, het open source is en het feit dat het alles lijkt te omvatten wat voor hen gewenst is. De opzet van deze bachelorproef is om te onderzoeken wat de mogelijkheden met elastic stack zijn omtrend proactief monitoring aan de hand van log files. 

Na een requiements analyse werden hun vereisten neergeschreven in het hoofdstuk casus.
Er werd onderzocht in welke mate Elastic stack aan hun vereisten voldeed door het ïmplementeren van een proof of concept. De proof of concept werd uitgewerkt voor twee besturingssystemen en twee databanken.

Dit werd uitgewerkt voor de drie core elementen en vier extra tools. In Logstash werd de input bekeken en hoe deze genormaliseerd kan worden. En er werd aandacht besteed aan het schrijven van een config file. Voor Elasticsearch wordt een beeld geschets van hoe de databank juist werkt hoe een zoekopdracht uit te voeren. Elasticsearch maakt het mogelijk om enkele tools erg efficiënt te laten werken. Deze tools bevatten de functionaliteiten die nodig waren voor deze casus. In de grafische interface Kibana kunnen dan beduidende grafieken gemaakt worden.
Om sommige vereisten aan te tonen waren ook theoritische berekeningen nodig.

Elastic stack voldeed aan elke voorwaarde al is verder onderzoek naar een goeie anomaly detectie methode wel aangeraden. Er kwamen reeds enkele mogelijkheden aanbod maar deze zijn nog voor verbetering vatbaar.
Elastic stack is een snel groeiend product en zal in de toekomst alleen maar meer mogelijkheden omvatten.
%%=============================================================================
%% Voorwoord
%%=============================================================================

\chapter*{Voorwoord}
\label{ch:voorwoord}
%% TODO:
%% Het voorwoord is het enige deel van de bachelorproef waar je vanuit je
%% eigen standpunt (``ik-vorm'') mag schrijven. Je kan hier bv. motiveren
%% waarom jij het onderwerp wil bespreken.
%% Vergeet ook niet te bedanken wie je geholpen/gesteund/... heeft
Deze is bachelorproef geschreven binnen het kader van het verplicht opleidingsonderdeel Bachelorproef voor het behalen van een diploma bachelor in de Toegepaste Informatica.

\textbf{Stageplaats}

Het onderwerp werd voorgesteld door mijn stagebedrijf. Daarom daarvoor zou ik graag mijn stageplaats oXya willen bedanken. Niet alleen voor het onderwerp maar ook de tijd die ze besteed hebben om mij te helpen in dit onderzoek. 
\begin{itemize}
	\item Dhr. Martin Van Den Abeele (co-promotor) - Bij hem kon ik altijd terecht met mijn technische vragen en hij bezorgde mij nuttige data. Dit zorgde voor een prettige samenwerking in een zeer leerrijk proces.
   
   
   \item Dhr. Lukas Becue - Hij ielp met het onderhouden van de elastic stack en het creëren van testdata.
\end{itemize}

\textbf{Familie}

Ik zou graag mijn ouders bedanken voor de steun, niet alleen nu maar doorheen heel mijn studies.
Ook voor hun bijdrage tot het verbeteren van deze bachelorproef, en verbeteren in de ruime zin van het woord.

\textbf{Docenten aan de HoGent}
	\begin{itemize}
	\item Dhr. Bert Van Vreeckem (promotor) - Hij volgde deze bachelorproef op en verzorgde nuttige feedback die leidde tot het eindresultaat.
   
   
   \item Dhr. Buysse Jens - Voor het in goede banen laten lopen van dit opleidingsonderdeel en de goede communicatie er rond.
\end{itemize}



%---------- Inhoudstafel ------------------------------------------------------
\pagestyle{empty} % No headers
\tableofcontents % Print the table of contents itself
\cleardoublepage % Forces the first chapter to start on an odd page so it's on the right
\pagestyle{fancy} % Print headers again

%---------- Lijst afkortingen, termen -----------------------------------------
%% Als je een lijst van afkortingen of termen wil toevoegen, dan hoort die
%% hier thuis. Gebruik bijvoorbeeld de ``glossaries'' package.

%%---------- Kern -------------------------------------------------------------

%%=============================================================================
%% Inleiding
%%=============================================================================

\chapter{Inleiding}
\label{ch:inleiding}

\section{De opdrachtgever, Oxya}
\label{sec:de-opdrachtgever}

Oxya is een internationaal bedrijf met onder meer een vesteging in Kortrijk. Het bedrijf houdt zich enkel en alleen bezig met het hosten van SAP systemen. Sap is een ERP pakket die gebruik gemaakt wordt voor de ondersteuning van bedrijfsprocessen. Bedrijven vanaf enkele werknemers gebruiken dit voor het bijhouden van financiële gegevens, voorraad aantallen, contactgegevend, \dots. Als de servers niet naar behoren functioner kan het voorkomen dat een heel bedrijf stil ligt tot de servers terug normaal functioneren. Het is dus hun functie om te monitoren als alles werkt zoals het hoort. Daarvoor zijn ze steeds opzoek naar nieuwe manieren om hen hierbij te helpen.

Ze hebben reeds twee tools ter beschikking namelijk PRTG en een eigen ontwikkelde tool koaly. Beide tools vragen systeem gegeven op en gaan dan kijken als de waarden niet te hoog of te laag zijn. Om te kijken als de waarden niet te hoog of te laag zijn zijn limieten vast gelegd en deze worden thresholds genoemd. Deze limieten staan vast en moeten dus handmatig gewijzigd worden indien nodig. Dus momenteel maakt oXya enkel gebruik van statische thresholds in hun tools.

\section{Probleemstelling en Onderzoeksvragen}
\label{sec:onderzoeksvragen}

De vraag voor dit project komt dus vanuit het bedrijf Oxya. De concrete vraag staat beschreven in \ref{ch:casus}. De opzet van deze bachelorproef is nagaan als elastic stack een juist keuze is voor deze casus.  

Deze bachelorproef zal een antwoord zoeken voor enkele onderzoeksvragen die werden opgesteld aan de hand van de casus.

\begin{itemize}
	\item Kan het opzetten en gebruiken van elastic stack zonder voorkennis op een relatief korte termijn aangeleerd worden?

	\item Kan elastic stack autonoom werken na de initializatie fase?
    
    \item Wat zijn op korte en lange termijn de gevolgen voor de hardware als elatic stack geïmplementeerd wordt?

	\item Bezit elastic stack de mogelijkheid anomalieën te detecteren?
    
    \item Is het mogelijk live duidelijk alerts te genereren? 

	\item Hoe goed scoort deze oplossing voor de beschreven casus?
\end{itemize}


\section{Literatuur studie}
\label{sec:literatuur-studie}

Elastic stack is een open source project waarvan alle bron code terug te vinden is op github. 
Elastic stack bestaat uit drie core elementen: Logstash, Elasticsearch en Kibana. Deze drie elementen werden samen gebracht tot Elastic stack en werken nu op een erg vlotte manier samen. Elastic stack heeft nu ook tools en deze zijn samen gebracht in X-Pack. Er zijn vier tools die besproken zullen worden namelijk: watcher, shield, monitoring en machine learning. Deze zijn gekozen omdat ze potentieel mogelijkheden hebben met de data die gegenereerd zal worden uit de log files.
Elk van de deze programma's is geschreven in java en beschikt over uitgebreide documentatie \autocite{documentatiesite}. Binnen de Elastic stack wordt dan weer gebruik gemaakt van zelf ontwikkelde talen. Ook deze beschikken over de nodige documentatie die via de Elastic site te verkrijgen is. 
Op dit moment zijn met uitzonderingen geen stukken voorbeeldcode te vinden. De oorzaak hiervoor is dat de elastic stack nog volop aan het groeien is. Dit is te danken aan de populariteit van Elastic stack die snel toe neemt. Zo is er een sprong gemaakt van versie 3 naar versie 5 waar heel wat basis functionaliteiten gewijzigd zijn. In het verloop van deze bachelorproef zal gewerkt worden met versie 5.2 .

Bedrijven als eBay, Netflix, The Guardian, \dots gebruiken elastic stack voor het beheren van hun data \autocite{15companies}. Dit wijst er dus op dat het geschikt is om te werken met terabytes data. Dit zijn bedrijven die hun code liever niet delen. Er is dus niet geweten in welke mate ze gebruik maken van elastic stack en welke uitbreidingen zij eventueel gemaakt hebben.

Aangezien de voorbeelden online beperkt zijn zal de grootste bron van informatie de documentatie van Elastic zelf zijn. Ook het aantal boeken is eerder beperkt. Het grootste probleem is dat door de snelle evolutie van elastic stack heel wat documentatie outdated is, zeker na de sprong van versie 3 naar 5. Gelukkig is er wel een actief forum\footnote{https://discuss.elastic.co/} waarop je meestal antwoord krijgt binnen de paar uur.

\section{Opzet van deze bachelorproef}
\label{sec:opzet-bachelorproef}

De rest van deze bachelorproef is als volgt opgebouwd:

Eerst en vooral zal de casus geschets worden in \ref{ch:casus}. Daarna zal in \ref{ch:elasticstack} aan  de hand en overzicht de globale werking van Elastic stack uitgelegd worden. 
In de daarop volgende hoofdstukken \ref{ch:logstash},\ref{ch:elasticsearch-xpack} en \ref{ch:kibana} wordt de werking van elke core component van elastic stack toegelicht. Er zal ook telkens onmiddelijk gekeken worden welke mogelijkheden dit met zich mee brengt voor onze casus. 
\ref{ch:evaluatie-casus} Bevat de evalutie van alle onderzoeksvragen die uit de casus gehaald werden.
In Hoofdstuk \ref{ch:conclusie}, tenslotte, wordt de conclusie gegeven en een antwoord geformuleerd op de vraag als elastic stack een meerwaarde kan hebben voor oXya. Daarbij wordt ook een aanzet gegeven voor toekomstig onderzoek binnen dit domein. Ook zal de mening van Oxya gevraagd worden omtrend de bruikbaarheid van elastic stack voor monitoring. 
De installatie zal niet aan bod komen omdat deze reeds heel goed uitgelegd is op de elastic download pagina.
%%=============================================================================
%% Elastic stack
%%=============================================================================

\chapter{Elastic stack}
\label{ch:elasticstack}

% BVV: Ik zou dit hoofdstuk voor de bespreking van de onderdelen plaatsen. Het is interessant als je lezers al een algemeen beeld hebben van de stack voordat je je verdiept in de componenten.

\section{Introductie}
Hier schetsen hoe elastic stack in elkaar zit.
%%=============================================================================
%% Casus
%%=============================================================================

\chapter{Casus}
\label{ch:casus}
Zoals steeds is oXya opzoek naar nieuwe manieren om nog beter te kunnen monitoren. Een aanvulling op hun monitoringtools zou dus het gebruiken van log files kunnen zijn. Deze files worden enkel bekeken als er zich een probleem voordeed om uit te zoeken wat misgelopen is.
Bij oXya ontstond zo de vraag als het niet mogelijk was om deze files te gaan monitoren en zo sneller problemen te zien komen.
Na wat research vonden ze dat Elastic stack in opmars is. Op het eerste gezicht heeft het de functionaliteiten waarnaar ze opzoek zijn. De concrete vraag is dus het ontwikkelen van een monitoringtool die gebruik maakt van Elastic stack om hun log files te analyseren.
In de volgende alinea's zal beschreven worden wat belangrijk is voor oXya.

Bij oXya is het gebruikelijk dat elke werknemer kennis heeft van alles. Ze verwachten dus dat deze zowel kennis heeft van Linux, Windows, SAP, oracle, \dots. Dus om de werknemers niet nog extra te belasten, is het belangrijk dat het gebruik van de tool makkelijk aan te leren is. 
Een tool die makkelijk is in gebruik zal dan ook vaker gebruikt worden. Deze bachelorproef zal dan ook zo geschreven worden dat deze gebruikt kan worden voor het aanleren van enkele basismogelijkheden met Elastic stack.

Het is ook belangrijk dat de tool autonoom kan werken. Het is dus niet de bedoeling dat iemand op een knop moet drukken om te zien als er meldingen zijn. Maar wel dat deze zich automatisch live meldt als ze zich voordoen. 

De implementatie mag ook maar beperkte negatieve gevolgen hebben voor de klant. Zo kan niet verwacht worden dat de klant zijn servers moet upgraden. Er zal dus bekeken worden wat de impact is voor een klant zodat oXya de afweging kan maken.

Oxya werkt met allerlei servers, platformen en databases, die elk hun eigen log files en alertfiles hebben en deze hebben ook elk hun eigen formaat of record layout. Vermits dit een groot volume aan data kan vertegenwoordigen, moet er ook gefilterd worden in de data. Het moet dus mogelijk zijn om alles te normaliseren en dan samen te brengen tot één geheel. Ook het uitbreiden moet mogelijk zijn indien ze in de toekomst een nieuw programma willen gebruiken.

Verwacht wordt dat het programma een probleem ziet aankomen \autocite{anomalydetection}. Dit kan door het herkennen van bepaalde errorcodes in de logs of door een stijging in een bepaald aantal errors of het aantal logs. Dus elke afwijking van het normaal gedrag, ook wel anomalie genaamd, moet gemeld worden. Enkele mogelijkheden hiervoor zullen dus bekeken worden en geëvalueerd.

Als een probleem zich zou voordoen, wordt verwacht dat er een duidelijk alert opgesteld wordt. Het alert moet weergeven wat het probleem juist is en wat de reden is. Aangezien een klant over meerdere servers beschikt, is het ook belangrijk aan te geven op welke server het probleem zich juist bevindt. Het is ook de bedoeling dat er geen overbodige alerts gegenereed worden, want anders zal geen aandacht meer besteed worden aan deze alerts.

%%=============================================================================
%% Logstash
%%=============================================================================

\chapter{Logstash}
\label{ch:logstash}


\section{Introductie}
\label{sec:logstash-introductie}

Logstash is een open source programma die data verzamelt. Het werkt real-time en zal dus reageren van zodra er nieuwe data beschikbaar is. Het heeft een vijftigtal input mogelijkheden zoals: databanken, grafieken, files, live webpagina's, \dots. In deze bachelorproef zal slechts gebruikt gemaakt worden van één optie namelijk files, meer specifiek log files. Dus als verder in deze bachelorproef gesproken wordt over data dan gaat dit strikt over de data die in log files gevonden kan worden.
Logstash beschik over de mogelijkheid de data te normalizeren. Normalizeren is het omzetten van iets naar een standaard formaat. Met logstash zal dus gezorgd worden dat vergelijkbare elementen gemaakt zullen worden waar later mee gewerkt kan worden.
Logstash kan vanuit verschillende bronnen tergelijk inlezen en verwerken. 
De kracht van Logstash zit hem erin dat het real-time werkt. Dit is zeer belangrijk voor de opzet van deze paper aangezien het de bedoeling is dat er onmiddellijk een alert ontstaat van zodra is misloopt. Het beschikt ook over enkele functies om de data te verwerken voorbeelden hiervan zijn: mix, match,filter, … .  
Het beschikt ook over meer dan 200 plugins en een grote community.
Het doel van Logstash is voornamelijk het normalizeren van data. Dit is zeer belangrijk voor later in elasticsearch. Elke lijn uit een log file zal gematched worden aan een patroon. Dit patroon zal zo enkele vaste indeces vast leggen waar later mee gewerkt zal worden.
Al deze regels worden vast gelegd in een config file. Per type log zal dus een nieuwe config file geschreven moeten worden. Voor deze paper zal het beperkt worden tot de syslogs van een linux machine en de logs van een oracle database.
Logstash werkt in drie fases: input,filter,output. Bij input zal mee gegeven worden welke files juist in de gaten gehouden zullen worden. Bij de filter zal de data verwerkt worden, dit gebeurd lijn per lijn. Elke lijn wordt beschouwd al een nieuw element. In de filter gebeuren de nodige bewerkingen op een lijn om zo te matchen met enkele indices. Deze indices zullen dan samen gebracht worden en vormen per lijn een JSON element. In het output deel wordt beslist wat met de JSON elementen moet gebeuren.  
Dit is enkel een korte beschrijving val de drie delen maar ze beschikken over nog enkele mogelijkheden. Enkele van deze extra mogelijkheden zullen nog aan bod komen wat verder in deze paper. 
Logstash draait standaard op poort 9600. Maar van zodra meer dan één config file gebruikt wordt stijgt het poort nummer telkens met 1. 



\section{config files}
\label{sec:logstash-config-files}

Voor elk type logs moet een config file ontwikkeld worden. Deze config files moeten aan een vast patroon voldoen. Dit vast patroon bestaat uit drie delen en deze zullen hieronder uitgediept worden. 
Logstash beschikt reeds over heel wat functionaliteiten en deze kunnen hier niet allemaal besproken worden. Er zal beperkt worden tot de meest relevante functionaliteiten met betrekking tot dit onderzoek. Een config file voldoet aan de normale filename conventies en heeft de extensie “.conf”.
Logstash heeft ook enkele voor geprogrammeerde patronen te gebruiken zijn binnen de volledige config file \autocite{logstashpatterns}. 

De vier config files geschreven in functie van deze bachelorproef zijn terug te vinden in \ref{ch:appendix}.

\section{Input}
\label{sec:logstash-input}

In \ref{subsec:lokaal} en\ref{subsec:beats} zal besproken worden hoe we files kunnen inlezen. Een combinatie van input streams is mogelijk. Deze files worden normaal lijn per lijn gelezen en verwerkt. Indien dit niet gewenst is kan dit aangepast worden naar wens. Dit zal besproken worden in \ref{subsec:multiline}.

\subsection{Lokaal}
\label{subsec:lokaal}

Het simpelste is als een file lokaal te vinden is. Dit houd dus in dat  de file op dezelfde (virtuele) machine te vinden is als waarop de logstash geïnstalleerd werd. Er dient dan wel gezorgt te worden voor een account die read rechten heeft voor deze file. Als meerdere files aan hetzelfde format voldoen kunnen deze allemaal samen ingelezen worden.
Het is belangrijk dat het absolut path gegeven word. 
\lstset{escapechar=@,style=customc}        
\begin{lstlisting}[frame=single]  
input {
	file {
		path => "/var/log/messages"
		path => "/var/log/auth.log"
		path => "/var/log/kern.log"
		path => "/var/log/cron.log"
	}
}
\end{lstlisting}

\subsection{Beats}
\label{subsec:beats}

Beats is een programma die nu ook tot de Elastic stack behoort. Het wordt niet beschouwd als core maar eerder als tool. Het is een licht programma die nieuwe data uit een file real-time via het netwerk naar een andere machine kan sturen. Dit gebeurd via een open netwerk poort op die andere machine waarop logstash staat. Logstash is op hetzelfde moment aan het luisteren naar die poort en kan zo de nieuwe data verkrijgen en dan ook verwerken.
Ook moet voor de configuratie van de beats het ip adres en de open poort van de andere machine gekent zijn. . In het volgend voorbeeld zal hiervoor de nog niet gebruikte poort  5043 gebruikt worden. En het ip adres is 192.168.0.100.
\begin{lstlisting}[frame=single]  
input {
	beats {
		port => "5043"
	}
}

\end{lstlisting}
Dit is het stuk code die in de config file hoort op de machine waarop logstash staat.
\begin{lstlisting}[frame=single]  
filebeat.prospectors:
- input_type: log
	paths:
		- /var/log/messages 
		
output.logstash:
	hosts: ["192.168.0.100:5043"]
\end{lstlisting}
Dit is het stuk code die in de beats config file hoort .


\subsection{Multiline}
\label{subsec:multiline}

Multiline is een plugin en moet dus nog geïnstalleerd worden als deze voor de eerste maal gebruikt wordt. Als u zich in de folder van logstash bevindt is het commando hiervoor:

\begin{lstlisting}[frame=single]  
bin/logstash-plugin install logstash-filter-multiline
\end{lstlisting}

Multiline kan ervoor zorgen dat een file niet lijn per lijn gelezen wordt maar dat hij bepaalde lijnen samen in leest. Dit is nodig voor onze oracle logs, soms bestaat één message uit meerdere lijnen.
Er kan een patroon opgegeven worden waaraan lijnen wel of net niet moeten voldoen om afzonderlijk te kunnen bestaan. Er kan mee gegeven worden wat er moet gebeuren met een lijne die niet aan de wensen voldoet. Als deze aan de lijn er voor of erna toegevoegd moet worden. 
Een probleem die opdook was dat logstash nooit de laatste lijn las. Dit was omdat lijnen die niet alleen konden bestaan aan de lijn ervoor toegevoegd werden. Logstash kon dus pas een message inlezen als hij zeker was dat ze volledig was. Omdat de log message altijd in één keer geschreven wordt kan ervanuit gegaan worden dat de message compleet is. Daarom werd een auto-flush geïmplementeerd die als er binnen de seconde geen nieuwe lijn komt de message als compleet zal beschouwen.

\lstset{escapechar=@,style=customc}  
\begin{lstlisting}[frame=single]  
input {
	...
	codec => multiline {
		pattern => "%{DAY} %{MONTH} %{MONTHDAY} %{TIME} %{YEAR}"
		negate => true
		what => "previous"
		auto_flush_interval => 1
	}
}
\end{lstlisting}

In pattern wordt het gewenste patroon geplaats. Dit doormiddel van een regex of de voorgeprogrammeerde patronen. In negate wordt aangegeven als het patroon wel of net niet moet matchen. De what geeft hier aan dat de lijnen die niet voldoen aan het patroon aan de vorige lijn toegevoegd zullen worden. En de auto-flush staat op één seconde om de delay te beperkten.


\subsection{Windows eventlogs}
\label{subsec:windows-eventlogs}

Windows vergt een andere manier van werken voor het verkrijgen van de eventlogs. Om deze te verkrijgen hoeft geen pad opgegeven te worden maar wel het type en welke eventlog gewenst is. 

\lstset{escapechar=@,style=customc}  
\begin{lstlisting}[frame=single]  
input {
  eventlog {
    type  => 'Win32-EventLog'
    logfile  => 'Application'
  }
}
\end{lstlisting}

\section{Filter}
\label{sec:filter}

Hier gebeuren alle bewerkingen en zal bepaald worden wat in de JSON gaat en wat niet. In de filter is het de bedoeling dat de file zoveel mogelijk genormaliseerd wordt. Een log lijn wordt eerst volledig ingelezen in een field  genaamd “message”. Van daaruit is het de bedoeling dat vast voorkomende stukken tekst eruit gehaald worden en in een eigen veld geplaatst worden, indien deze later gebruikt willen worden.  Om deze speciefieke velden te maken zijn enkele mogelijkheden deze zullen besproken worden in \ref{subsec:patroon} en \ref{subsec:zoeken}. Werken met de tijd is niet altijd even simpel en zal daarom afzonderlijk besproken worden in \ref{subsec:timestamp}. Er zal afgesloten worden in\ref{subsec:overige} met nog enkele nuttige commando’s. 
Hou er rekening mee dat elasticsearch zal zoeken aan de hand van de fields die hier gemaakt worden. De juiste field keuzes kunnen dus een grote invloed hebben op de zoeksnelheid.


\subsection{Patroon}
\label{subsec:patroon}

Om een file te normaliseren zal gebruik gemaakt worden van een patroon. Aan de hand van dit patroon worden enkele fields gemaakt. Belangrijk is dus dat waarden in een log steeds op dezelfde plaats te vinden zijn.
Indien dit het geval is kan hier het meeste werk van de config file gebeuren. Hiervoor zal een regex opgesteld worden die gebruik maakt van de voorgeprogrammeerde patronen. 
Met \%{TYPE:name} kan een field name gemaakt worden met die voldoet aan het voorgeprogrammeerde type. Op het einde van de logs die in deze casus gebruikt worden staat op het einde van de lijn een boodschap.
Om deze boodschap ook in een field te krijgen werd gebruik gemaakt van een tweede mogelijkheid. Met (?<name>regex) kan weer een field name gemaakt worden die voldoet aan de regex binnen de haakjes.

\lstset{escapechar=@,style=customc}  
\begin{lstlisting}[frame=single]  
filter {
	grok {
		match {
			"message" => "%{MONTH:month} +%{MONTHDAY:monthday} %{TIME:time} %{NOTSPACE:user}(?<log_message>.*$)"	 
		}
	}
}
\end{lstlisting}

\subsection{Zoeken}
\label{subsec:zoeken}

Sommige programma’s hebben een eigen type errors of messages. Deze message zijn opgebouwd op een vast patroon ook. Dit maakt het mogelijk om met logstash te zoeken naar dit patroon en op die manier deze messages in een field te steken. De error messages van oracle kunnen hier als voorbeeld dienen. Deze message zijn opgebouwd uit “ORA-“ gevolgd door vijd cijfers. Er zal dus gezocht worden als dit patroon in de message te vinden is en als dat het geval is zal deze in een field geplaatst worden.

\lstset{escapechar=@,style=customc}  
\begin{lstlisting}[frame=single]  
filter {
	if [message] =~ /ORA-/ {
		grok {
			match => [ "message","(?<ORA->ORA-[0-9]*)" ]
		}  
	}
}
\end{lstlisting}

\subsection{Timestamp}
\label{subsec:timestamp}

Logstash zal altijd als standaard timestamp het moment nemen waarop hij de data verwerkt. Als dit live gelezen wordt is dit goed maar als er ook logs van vroeger willen inladen worden geeft dit problemen.  Daarom zal meestal de tijdsaanduiding in het begin van een log lijn gebruikt worden als timestamp. Als voorbeeld het jaar niet gegeven werd zal logstash automatisch het jaar nemen waarin het zich bevindt. Er wordt nu vanuit gegaan dat de tijdsaanduiding gecapteerd kan worden aan de hand van het patroon die uitgelegd werd in \ref{subsec:patroon}. Eerst wordt een tijdelijk field gemaakt met alleen de tijdsaanduidingen. Daarna kan de tijdzone gekozen worden en de timestamp gelijk gezet worden met ons tijdelijk field. Er moet wel mee gegeven worden hoe het tijdelijk field opgebouwd is. 

\lstset{escapechar=@,style=customc}  
\begin{lstlisting}[frame=single]  
filter {
	...
	mutate {
		add_field => {
			"timestamp" => "%{year} %{month} %{monthday} %{time}"
		}
	}

	date {
		timezone => "CET"
		match => [ "timestamp","yyyy MMM dd HH:mm:ss" ]  
	}
}
\end{lstlisting}

\subsection{Overige}
\label{subsec:overige}

Enkele nuttige commado’s die nog niet aan bod kwamen zijn het kopiëren en het verwijderen van fields. Zo kan het voorbeeld handig zijn om message op het einde te vervangen door de log message alleen zonder alle andere info. Deze info zit dan toch al in andere field en zo kan de message makkelijker leesbaar gemaakt worden. Fields die gemaakt werden voor tijdelijk gebruik kunnen best verwijderen worden. Dit om de simpele reden dat ze enkel extra geheugen in nemen en toch tot niets dienen.

\section{Output}
Voor de output zijn weer heel wat mogelijkheden. In deze paper zullen er slechts drie aanbod komen. Hier zal dus beschreven worden wat er moet gebeuren met de JSON elementen.

\subsection{Standaard output}
\label{subsec:standaardoutput}

Dit is vooral nuttig voor tijdens  de testfase. Als logstash dan gestart wordt vanuit de commandline kan de output live gevolgd worden. Dit heeft als voordeel dat onmiddellijk gezien kan worden als het werkt zonder errors. De code hiervoor is dan ook heel simpel.

\lstset{escapechar=@,style=customc}  
\begin{lstlisting}[frame=single]  
output {
	stdout {
		codec => rubydebug 
	}
}
\end{lstlisting}

\subsection{Elasticsearch}
\label{subsec:elasticsearch}

Door het samenbrengen van deze programma’s binnen de Elastic stack is het zeer makkelijk om de elementen door te geven. Zo kan in enkele lijnen duidelijk gemaakt worden dat alle JSON elementen naar Elasticsearch moeten. Om Elasticsearch efficiënt te laten werken wordt een index verwacht. Deze index wordt dan toegevoegd aan elk JSON element die hier gegenereerd werd. Aangeraden wordt om in eerste instantie het type logs te vermelden in de index voorbeeld: linux, Oracle, Windows, \dots. Als extra kan ook een combinatie gemaakt worden met de servernaam om zo duidelijk te maken waar de logs vandaan komen. 
Via hosts kan duidelijk  gemaakt worden waar elasticsearch juist aan het draaien is. In dit voorbeeld draait elasticsearch lokaal maar er kan ook een ip adres opgegeven worden.

\lstset{escapechar=@,style=customc}  
\begin{lstlisting}[frame=single]  
output {
	elasticsearch {
		hosts => ["localhost:9200"]
		index => "linux_citts"
	}
}
\end{lstlisting}

\subsection{Alerts}
\label{subsec:alerts}

Het is ook mogelijk alerting te doen binnen logstash. In deze paper zal dit gebruikt worden om te alerten op belangrijke  “ORA” errors. Als één van de errors voorkomt wordt deze in het field “ORA-” geplaatst (zie stap \ref{subsec:zoeken}). Dit voorbeeld zal kijken als er zich geen invalid state voordoet. De gekende errors voor invalide state zijn: ORA-01502 en ORA-20000. Als dus één van deze errors zich voordoet zal een email met een alert gestuurd worden. In de body wordt enige duiding gegeven rond waar en wanneer de error zich voor deed.

\lstset{escapechar=@,style=customc}  
\begin{lstlisting}[frame=single]
output {
	if "ORA-01502" == [ORA-] or "ORA-20000" == [ORA-] {
		email{
			subject => "Invalid State"
			to => "bvervaele@oxya.com"
			body => "Host: %{host}\n\nTime: %{@timestamp}\n\nLine of the error: %{message}"
			address => "smtp.gmail.com"
			port => 587
			username => "logserviceoxya@gmail.com"
			password => "oxya1234"
			use_tls => true
		}
	}	 
}

\end{lstlisting}
%%=============================================================================
%% Elasticsearch
%%=============================================================================

\chapter{Elasticsearch}
\label{ch:elasticsearch}

\section{Introductie}
\label{sec:introductie}

Elasticsearch is het hart van elastic stack. Het is een zoek algoritme gebasseerd op indices. Voor een kleine hoeveelheid data wordt beter iets anders gebruikt aangezien de start tijd van elasticsearch} relatief lang is.
Maar voor grote hoeveelheden data is het zeer geschikt. Na de starttijd kan zo goed als live de zoekopdrachten volbrengen. 
Omwille van de goeie schaalbaarheid wordt het reeds bij enkele bedrijven gebruikt die met heel veel data af te rekenen krijgen **bron**.

Om elasticsearch optimaal te laten werken is het zeer belangrijk om juist om te gaan met indices. Als voorbeeld gezocht wil worden op een bepaalde dag van de week dan kan hiervoor best een afzonderlijk field voorzien worden. 
Daarom is het dus erg belangrijk om werk te maken van het schrijven van een goede config file voor logstash.

Het gebruik van de zelf ontwikkelde taal is niet eenvoudig en daarom zal in \hyperref[sec:search]{\ref{sec:search}} enkele basis mogelijkheden toegelicht worden.
De commando's kunnen via de console in kibana gerund worden of via de commandline. In de rest van de paper zullen de stukken code gebruikt worden voor de in de console van kibana.
Indien gewenst kunnen deze stukken code ook gebruikt worden in de commandline maar dan wel in combinatie met curl.

\section{Search}
\label{sec:search}

Komt nog

\section{Watcher}
\lable{sec:watcher}

Watcher is één van de tools die tot de elastic stack behoren. Het maakt het mogelijk om in bepaalde gevallen een alert te sturen.
Het bestaat uit vier delen: planning, zoekopdracht, conditie en actie. Dit is dus ook een erg belangrijke tool voor het monitoren.


%%=============================================================================
%% Kibana
%%=============================================================================

\chapter{Kibana}
\label{ch:kibana}

\section{Introductie}
\label{sec:introductie}
Kibana is het programma die voor de visualisatie binnen de elastic stack zorgt. Deze zal default te vinden zijn op poort 5601. 
De eerste keer dat kabina gestart wordt zal gevraagd worden voor een index pattern. Kibana kan dus pas gebruikt worden als er al data in elasticsearch zit.
Aan de hand van het index pattern kan al gefilterd worden. Het is ook mogelijk om een regex op te geven als index pattern. Dus * zal alle data inlezen en gebruiken. 
Als "Time-field name" zal in deze casus altijd "@timespamp" gebruikt worden. Deze is in logstash zo ingesteld om overeen te komen met de timestamp van de log zelf.

Eenmaal deze instelling gebeurd zijn kan kibana in gebruik genomen worden. Aan de linker kant bevinden zich enkele symbolen die alle opties weergeven. 
De opties die voor de casus van belang zijn zullen hieronder verduidelijkt worden.

\section{Discover}
\label{sec:discover}

Hier kunnen alle json elementen gevonden worden die matchen met het opgegeven index pattern. Bovenaan wordt een grafiek getoont van wanneer hoeveel logs gegenereerd zijn. 
Aan de linker kant wordt dan weer een lijst getoont met welke fields allemaal te vinden zijn binnen deze selectie van data. 
De voorstelling van de data ziet er chaotisch uit op het eerste zich. Deze kan eenvoudig gewijzigd worden door bij een element op het pijltje te drukken zodat alle fields voor dit element getoond worden.
Deze fields kunnen dan gebruikt worden in de tabel. Om een field als kolom vast te zetten in de tabel moet éénmaal gedrukt worden op het derde icoontje. 
Zo kan dus een overzichtelijke tabel bekomen en kan inzicht in de data verkregen worden.


\section{Visualize}
\label{sec:visualize}

Onder visualize valt een grote variatie aan dingen: grafieken, getallen, tekst, heatmap, \dots. Voor deze casus zal voornamelijk gebruik gemaakt worden van grafieken en getallen.
Het is belangrijk om de juist keuze te maken van visualisatie, zowel van de keuze voor de visualisatie als de data die gebruikt wordt er voor. 

Als gekozen wordt om een grafiek te maken is het belangrijk eerst te kiezen wat er op de assen moet komen. Indien gewenst kan deze grafiek nog verder onderverdeeld/gfilterd worden door sub-buckets toe te voegen.
Deze onderverdeling kan bijvoorbeeld gemaakt worden op de value van een field. Dan dient binnen de sub-bucket gekozen te worde voor Terms en dan het gewenste field.
Ook een filter wordt via sub-buckets toegevoegd. Een filter juist definiëren wordt besproken in \hyperref[sec:filters]{\ref{sec:filters}}



\section{Timelion}
\label{sec:timelion}

In timelion kunnen grafieken met elkaar vergeleken worden. Zo kan na gegaan worden hoe een grafiek er de vorige weken uitzag ten op zichte van die van deze week.
Voor de casus kan dit zeker van pas komen voor het ontdekken van abnormale afwijkingen. 

Er zijn al reeds heel wat mogelijkheden geïmplementeerd \autocite{timeliongithub}. 
Maar om nuttige grafieken te verkrijgen zal veelal een combinatie gemaakt moeten worden van enkele van deze mogelijkheden.

Helaas kan hier nog geen gemiddelde en standaard afwijking berekend worden van de zelfde momenten in de afgelopen weken.
Dit zou voor deze casus handiger zijn omdat er 's morgen pieken zullen zijn die op die manier niet voorspeld kunnen worden.

\section{Dashboard}
\label{sec:dashboard}

In dashboard kunnen alle visualisaties samengebracht worden. Hier kan dus een duidelijk overzicht van de data gecreeërd worden. De plaatsing en keuze van grafieken is natuurlijk wel van belang.
Om het overzichtelijk te houden kunnen afzonderelijke dashboards gemaakt worden. Zo kan voor deze casus voorbeeld een dashboard gemaakt worden voor SAP, oracle, windows, \dots.
De dashboards zijn dan ook een belangrijke informatie bron bij het zoeken naar een fout. Hier kan de zoektoch gestart worden om te kijken als er geen opvallende afwijkingen waren of om een idee te krijgen van wanneer iets fout liep.

\section{Filters}
\label{sec:filters}

Er is reeds al een filter gebeurd bij het kiezen van het index pattern. Maar er zijn nog heel wat mogelijkheden.
De filter die meest gebruikt zal worden is een filter op de tijd. De tijd kan rechts bovenaan gewijzigd worden. Zo kan er naar een specifiek moment in de tijd gegaan worden om te zien wat er gebeurd is.

Voor de rest van het filteren wordt je aangewezen op de zoekbalk bovenaan. Er kan simpelweg gezocht worden op een woord die voorkomt in een json element. Zo kan voorbeeld gezocht worden op een specifieke ORA error door in de zoekbalk de code in te geven.
In de zoekbalk kunnen ook combinaties van filters gebeuren.

Er is ook de mogelijkheid om te zoeken op de waarde van een field. Dit kan aan de hand van volgende query: fieldName:"waarde"

Niet elk json element heeft alle fields. Als men enkel de elementen wil zien met een field genaamd "code" kan gezocht worden op volgende query: $\_exists\_:"code"$.

Als iets net niet mag voorkomen in een zoek opdracht kan men er een "!" voor tpyen.




%%=============================================================================
%% Evaluatie casus
%%=============================================================================

\chapter{Evaluatie casus}
\label{ch:evaluatie-casus}

Hier zal bekeken worden als Elastic stack voldoet aan de vooropgestelde eisen in de casus.
Zowel de voor- als nadelen zullen aan bod komen en enkele eventuele alternatieven. Het is de bedoeling een duidelijk beeld te schetsen van de sterktes en zwaktes van Elastic stack.
Deze evaluatie gebeurde samen met Dhr. Van Den Abeele van oXya.

\section{Nodige kennis}
\label{sec:nodige-kennis}
	
Om dit te testen werd een werknemer van het bedrijf Oxya gevraagd enkele handelingen uit te voeren.
Hij kreeg daarvoor de hoofdstukken Logstash, Elasticsearch en X-Pack en Kibana ter beschikking en de mogelijkheid om via het internet te zoeken. 
Er werd ook telkens op voorhand een schatting gemaakt van de benodigde tijd.
De werknemer werd het volgende gevraagd:
\begin{enumerate}
    \item schrijf een config file voor logstash voor het lezen van een Linux syslog (2h),
    \item zoek in Elasticsearch op hoeveel logs er het laatste uur waren met het woord "sap"  in de message (30min),
    \item maak een grafiek die het aantal logs met een "word" field per tijd terug geeft (10min),
	\item intreperteer uit een gegeven dashboard wanneer iets is fout gelopen (2min). 
\end{enumerate}

1. 	De werknemer nam eerst snel het hoofsdstuk Logstash eens door. Daarna werd gekeken hoe hij de file kon normaliseren. 
Eenmaal dat beslist was, ging hij aan de slag met de input. Aangezien het path gegeven was, was de input correct geschreven op 2 minuten. Aangezien op voorhand goed nagedacht was hoe de normalisering moest gebeuren, ging het schrijven van het grok pattern ook vlot.
Als output werd std out gebruikt en Elasticsearch. Dit stond goed beschreven en was snel correct geschreven. 
Bij het eerste keer runnen van de config file werd gezien dat de datum nog niet juist was en er ook nog fields te veel waren. 
Het juist krijgen van de datum vergde nog wat extra tijd. Het verwijderen van de overbodige fields was dan weer heel snel gebeurd.

Om een volledig werkende config file te maken, zonder externe hulp, heeft het 1 uur en 6 minuten geduurd.

2. Ook hier ging de werknemer niet direct over tot de actie, maar nam eerst zijn tijd om bijpassend hoofdstuk door te nemen. 
Daarna heeft hij twee afzonderlijke queries geschreven die afzonderlijk bleken te werken na enkele pogingen. Na was prutsen met alle haakjes werden de twee queries samengebracht en dit gaf het gewenste resultaat in net geen 23 minuten.

3. Toen de werknemer niet direct een sectie zag over een grafiek maken, ging deze direct over naar de praktijk. Aangezien de symbolen duidelijk zijn, werd op geen tijd gevonden waar hij de juiste grafiek kon maken.
Een grafiek die het volledige aantal logs per tijd weergaf, ontstond in minder dan 5 minuten. Daarna kwam de moeilijkheid om enkel de logs te tonen met het "word" field. 
De werknemer werd nog enkele minuten gegeven waarin hij verschillende settings aanpaste. Na twaalf minuten was de gewenste grafiek nog niet in orde en werd de opdracht gestopt.

4. De werknemer kon na wat inzoomen op de grafieken tot op de minuut zeggen wanneer het was fout gelopen. Hij zei ook nog dat deze opdracht wel erg makkelijk was.
	

Hieruit wordt besloten dat het aanleren van de Elastic stack vlot verliep. Zeker de meest gebruikte stap, het gebruik van het dashboard, werd aangegeven makkelijk te zijn.
Vermeld moet wel worden dat gebruik gemaakt werd van enkele goed gekozen grafieken die problemen snel terug geven. 
Aangezien het maken van een speciale grafiek niet zo eenvoudig bleek te zijn werd een extra sectie geschreven. 
	
\section{Autonomie}
\label{sec:autonomie}	
	
De Elastic stack kan autonoom werken en zal alerts genereren zonder dat daarvoor iets extra hoeft gedaan te worden. 
Het is natuurlijk wel aan de eindgebruiker om te zorgen dat de grafieken en alerts relevant blijven. Dit zal dus nu en dan wat tijd vergen, maar dit zal eerder beperkt zijn. 
Na de ontwikkeling kan gesteld worden dat Elastic stack autonoom zal werken en een probleem live kan gaan alerten. 
	
	
\section{Hardware vereisten}
\label{sec:hardware-vereisten}
	
Belangrijk om mee te beginnen is dat na de installatie geen restart nodig is. Restarts worden zoveel mogelijk vermeden omdat de klant dan al zijn processen moet stoppen en dit voor overlast zorgt.

Aangezien deze bachelorproef slechts een proof of concept is, werd alles dan ook enkel geïmplementeerd op een testserver. Op deze testserver kon wel het cpu gebruik geobserveerd en geëvalueerd worden. 
Er bestaat een programma, Metricbeat genaamd, dat systeeminformatie opvraagt. Metricbeat kan op enkele minuten tijd met Elastic stack verbonden worden. Online kan een default dashboard gevonden worden die de systeem info weergeeft. Dit leek een heel eenvoudige manier om uit te vinden hoeveel cpu elk onderdeel gebruikte. Helaas valt alles onder de noemer java en wordt dan ook zo getoond.
In \ref{fig:java} werden alle java processen samen gemonitord voor 24 uur. 

\begin{figure}[h]
	\includegraphics[width=16cm]{img/java1}
	\caption{CPU gebruik van java over 24 uur.}
	\label{fig:java}
\end{figure}

Om dus te weten hoeveel nu echt naar de Elastic stack gaat, moet ook het cpu gebruik voor java gemeten worden als Elastic stack af staat.
Als Elastic stack uitgezet wordt, kan deze monitoring niet gebruikt worden. 
Daarom werden enkele steekproeven uitgevoerd en bleek dat java dan gemiddeld 1.7\% van de cpu gebruikt. Dus de Elastic stack gebruikt gemiddeld 2\% van de cpu op de testserver. 
De testserver is dan ook nog eens veel lichter dan de servers bij de klanten.
Dit zorgt ervoor dat een klant amper een verschil in cpu gebruik zal waarnemen door het implementeren van een Elastic stack.


Voor de disk space werden enkele berekeningen uitgevoerd. Als basis werd het gebruikte geheugen en het aantal elementen bekeken. Om dit te vinden gebruikten we volgend commando in de commandline:
\lstset{escapechar=@,style=customc}        
\begin{lstlisting}[frame=single]  
	curl -XGET 'http://localhost:9200/_nodes/stats'
\end{lstlisting}
Hieruit werd gevonden dat de testomgeving reeds 3502 elementen in de Elasticsearch heeft opgeslagen. Deze elementen zouden 1.968.548 bytes innemen.
Dit zou willen zeggen dat één JSON-lement dus ongeveer 562 bytes in beslag neemt.
De testomgeving bevat een voldoende groot aantal en variatie aan elementen om deze gegevens als representatief te beschouwen voor verdere berekeningen.
Daarna werd uitgezocht hoeveel elementen grote bedrijven als Stapels en Bekaert zouden hebben op één maand. Om dit te doen werden het aantal loglijnen geteld en gekeken over welke periode deze geproduceerd werden.
Indien de log slechts de laatste 24 uur bijhield werd een gemiddelde van drie verschillende dagen genomen. 

Staples (Linux)
\begin{itemize}
	\item syslog:  10.603 elementen/dag
	\item oracle: 280.322 elementen/40 dagen
\end{itemize}	

Bekaer (Windows)
\begin{itemize}
	\item syslog: 314.044 elementen/270 dagen
	\item MySQL:   30.891 elementen/175 dagen
\end{itemize}

Voor Staples zou dit betekenen dat er elke maand 528.331 elementen bijkomen. Dit komt overeen met 296.922.022 bytes of 297MB.
Bij bekaert zou een maand slechts 40.190 elementen bevatten. Dit komt dan weer overeen met 22.586.780 bytes of 22.5MB.

Er kan dus besloten worden dat, voor het beperkt aantal log files die voor deze casus bijgehouden worden, de nodige hoeveelheid disk space erg beperkt blijft. 
Ook valt op dat een windowsserver nog minder aan disk space zal in nemen dan een linuxserver.

\section{Flexiebel}
\label{sec:flexiebel}

oXya heeft geen vaste pakketten. Het maakt aanbiedingen op maat van elke klant. Hierdoor zitten ze dus met verschillende besturingssytemen en databanken.
Als besturingssystemen werden redhat en windows server 2012R2 getest. En als databases werden oracle en MySQL bekeken.
De config files zijn terug te vinden in \ref{ch:appendix}. 
Deze config files kunnen dus voor elk type logfile geschreven worden. Er is wel een verschil in hoe ver een file genormaliseerd kan worden.
Een oracle file zal voorbeeld verder genormaliseerd kunnen worden dan een MySQL file. Dit zal voor MySQL de monitoringmogelijkheden beperken.
Maar er kan dus gesteld worden dat het flexibel genoeg is aangezien het eender welke log files kan gebruiken als input data.


\section{Annomaly detectie}
\label{sec:annomaly detectie}

oXya zou graag op de hoogte gebracht worden als er zich ergens uitzonderlijk gedrag voor doet. De mogelijkheid is er om statische thresholds te plaatsen.
Maar in plaats daarvan wordt gekozen om te werken met dynamische thresholds. Dit om te verkomen dat bij een groeiend bedrijf telkens de waarden aangepast moeten worden.
In \ref{subsec:timelion} zal bekeken worden wat de mogelijkheden zijn met de vrijheid die daar gegeven wordt.
In \ref{subsec:machine-learning} daarentegen is de vrijheid beperkt, maar indien het van genoeg input voorzien wordt, is het zeker betrouwbaar.

De servers die oXya monitort zijn meer belast in de week dan in het weekend. Ook overdag is er meer activiteit dan 's nachts. Het ideale zou dus zijn om te gaan vergelijken met hetzelfde moment in de voorbije weken \autocite{hourweekanalyse}.

Over het algemeen beschikt Elastic stack dus over enkele nuttige anomaliedetectie mogelijkheden. 
Er is een breed aanbod en afhankelijk van de toepassing zal dus een gepaste anomaliedetectie gekozen moeten worden.

\subsection{Timelion}
\label{subsec:timelion}

Eén van de mogelijkheden is om de gegevens van dit moment te gaan vergelijken met het gemiddelde $\pm$ 3 keer de standaard deviatie van de voorbije waarden.
Dit zou willen zeggen dat met een 99,7\% zekerheid gezegd zou kunnen worden als een waarde uitzonderlijk is of niet. Om het gemiddelde en de standaard deviatie te berekenen zullen de waarden van de laatste 24 uur gebruikt worden. 
In deze situatie worden beste resultaten verkregen als de waarden per uur gegroepeerd worden. Om deze grafiek te creëren dienen volgende commando's uitgevoerd te worden:
\lstset{escapechar=@,style=customc}        
\begin{lstlisting}[frame=single]  
	.es(index="mssql").label("Values"),
	.es(index="mssql",offset=-1h).movingaverage(10).max(15).sum(.es(index="mssql",offset=-1h).movingstd(10).multiply(3)).label("gem + 3std"),
	.es(index="mssql",offset=-1h).movingaverage(10).max(15).sum(.es(index="mssql",offset=-1h).movingstd(10).multiply(-3)).label("gem - 3std")
\end{lstlisting}

\begin{figure}[h]
	\includegraphics[width=16cm]{img/avg3std}
	\caption{Anomaliedetectie aan de hand van het gemiddelde en de standaard deviatie.}
	\label{fig:avg3std}
\end{figure}

De grafiek van dit moment moet zich dus tussen de andere twee grafieken bevinden anders dient er een alert opgesteld te worden voor een anomaly.


Een andere mogelijkheid met timelion is de implementatie van het holt-winter algoritme \autocite{holtwinters}.
Ook dit heeft zijn voor- en nadelen. Het zal meer rekening gaan houden met de trend van de grafiek. Het enige probleem hiermee is dat er nog geen betrouwbaarheidsintervalen geïmplementeerd zijn.
Wat wel een mogelijkheid zou zijn is dat de grafiek zich tussen 0,5 en 1,5 keer de verwachte holt-winter waarde moet bevinden.
\lstset{escapechar=@,style=customc}        
\begin{lstlisting}[frame=single]  
	.es(index="mssql").label("Values"),
	.es(index="mssql",offset=-1h).holt(0.1,0.1).multiply(0.5).label("0.5*holt"),
	.es(index="mssql",offset=-1h).holt(0.1,0.1).multiply(1.5).label("1.5*holt")
\end{lstlisting}

\begin{figure}[h]
	\includegraphics[width=16cm]{img/holtwinters}
	\caption{Anomalie detectie aan de hand van Holt Winters.}
	\label{fig:holtwinters}
\end{figure}


Maar ook hier is het weer mogelijk om andere dingen mee te doen door een combinatie te maken met andere mogelijkheden.
Hier zijn dus heel wat mogelijkheden en de uitkomst zal afhangen van het gekozen algoritme. 

\subsection{Machine Learning}
\label{subsec:machine-learning}

Zoals reeds besproken is er bij Machine Learning, naast het kiezen uit de vier warnings, geen vrijheid. Maar dit hoeft ook niet, want als de juiste elementen hier als input gebruikt worden doet Machine Learning de rest. Hoe meer input Machine Learning krijgt hoe accurater er een alert gestuurd kan worden. 
Deze tool is nog in beta en heeft dus ook enkele kinderziektes. Zo werd ontdekt dat na het ontstaan van een vijftal anomanlieën op een termijn van 24 uur Machine Learning het moeilijk heeft om terug naar een stabiele toestand te gaan. Zo zou het in de daaropvolgende week mogelijk zijn dat nog grotere anomalieën niet opgemerkt worden.

Machine Learning heeft dus nog enkele minpunten, maar deze wegen niet op tegen de toch al nuttige functionaliteit.

\subsection{Eigen detectie}
\label{subsec:eingen-detectie}

Indien bovenstaande tools nog niet aan de wensen voldoen, is er nog altijd de optie om een eigen detectie tool te ontwerpen. Het ontwikkelen van een eigen detectie tool is mogelijk door gebruik te maken van de java api \autocite{javaapi}. 
Bij het ontwikkelen van een eigen detectie tool in java kan dan gebruik gemaakt worden van alle Elasticsearch queries. 
De java api zorgt ervoor dat de limiet van anomalie detectie bij jezelf ligt.

\section{Alers}
\label{sec:alerts}

De alerts die in Logstash gegenereed worden bevatten in de titel het probleem en op welke host het probleem zich stelt.
Als de e-mail dan geopend wordt, kan men terug vinden wanneer het probleem zich voor deed. Nog eens op welke machine het probleem vastgesteld werd. En dan de boodschap die bij het probleem hoort. 
Deze boodschap is voor een werknemer van oxya voldoende om te kunnen beslissen als er opvolging nodig is of niet.

Ook de alerting via Watcher kan duidelijk verlopen. Watcher biedt ook een breed gamma aan mogelijkheden aan. Door oXya werd gekozen om de alerting via email te laten verlopen.
Watcher maakt het mogelijk om te alerten op Elasticsearch queries. Maar Watcher kan ook gebruikt worden om te alerten op Timelion of op de Machine Learning.

Dus ook op gebied van alerting zijn er genoeg mogelijkheden.

%%=============================================================================
%% Conclusie
%%=============================================================================

\chapter{Conclusie}
\label{ch:conclusie}

Elastic stack voldoet aan alle vereisten vooropgesteld door oXya. Indien beslist zou worden om er mee verder te gaan, zal een analyse gemaakt moeten worden van welke andere log files nog gebruikt kunnen worden. Voor deze files zal dan een Logstash config file geschreven moeten worden. 
Dit zorgt er voor dat de implementatie wel wat werk zal vergen, maar daarna werkt het volledig autonoom. Elastic stack zou er dan voor zorgen dat sommige problemen vroegtijdig gemeld kunnen worden. Sommige tools en het dashboard kunnen dan geraadpleegd worden bij het onderzoeken van het probleem.

Naar de toekomst toe zijn dus heel wat mogelijkheden. Elastic stack is ook nog steeds erg snel aan het groeien en zal dus nog meer mogelijkheden bieden in de toekomst. Voor oXya is dit, volgens de geschetste casus, een gepast product. Maar de uitdaging zal liggen in het op de gepaste tijd zenden van meldingen.
Een mogelijk vervolg onderzoek zou kunnen zijn naar de anomaly detectie.
Een andere mogelijkheid zou zijn naar het zelf schrijven van elastic aangezien het open scource is en een java api heeft.

Hieronder wordt de mening van oXya zelf gegeven:


	``De eerste resultaten bij het gebruik van Elastic stack om een bijkomende vorm van monitoring te hebben, lijken zeer positief. 
	De mogelijkheid om verschillenden log files van uiteenlopende systemen (os, db , sap) te analyseren en daarin bepaalde kritische alerts te detecteren en te escaleren is zeer bruikbaar in onze omgeving. 
	Het online oppikken van alerts was een optie die te verwachten was van deze tool. 
	Maar wat nog interessanter blijkt te zijn, is het gedrag van bepaalde loglijnen in log files te laten analyseren en te laten interpreteren met de “Timelion” en de “Machine Learning“ modules.
	Wat de mogelijkheid geeft om statische beslissingen te treffen, tijdens het analyseren van deze (soms seasonal) data. 
	Bijvoorbeeld het gebruik van de "Seasonal Holt-Winters" methode om abnormaal gedrag te voorspellen, lijken duidelijk een stap in de goede richting. 
	Waar we vroeger enkel triggers gebruikten om alerts te generen, kunnen we nu ook anomalieën vroegtijdig detecteren. 
	Gezien er nog veel mogelijkheden te onderzoeken zijn in het domein van machine learning zijn we zeker van plan nog verder onderzoek in te plannen op dit terrein.''
	\begin{flushright} 
		-Van Den Abeele, Martin
    \end{flushright}



%%---------- Back matter ------------------------------------------------------

\printbibliography
\addcontentsline{toc}{chapter}{\textcolor{maincolor}{\IfLanguageName{dutch}{Bibliografie}{Bibliography}}}

\appendix
\chapter{Logstash config files}
\label{ch:appendix}

Linux:
\lstset{escapechar=@,style=customc}        
\begin{lstlisting}[frame=single]  
input {
    file {
        path => "/var/log/messages"
	path => "/var/log/auth.log"
	path => "/var/log/kern.log"
	path => "/var/log/cron.log"
    }
}
 
filter {
    
    grok {
        match => {
            "message" => "%{MONTH:month} +%{MONTHDAY:monthday} %{TIME:time} %{NOTSPACE:user}(?<log_message>.*$)"
        }
    }
 
    if [message] =~ /SAPFDG_00/ {
        grok {
            match => [ "message","(?<tcode>[A-Z]([A-Z]|[0-9])[0-9])" ]
        }  
    } 
    if [message] =~ /dump/ {
        grok {
            match => [ "message","(?<word>dump)" ]
        }  
    } 
    if [message] =~ /semaphore/ {
        grok {
            match => [ "message","(?<word>semaphore)" ]
        }  
    } 
    if [message] =~ /LOAD_NO_ROLL/ {
        grok {
            match => [ "message","(?<word>LOAD_NO_ROLL)" ]
        }  
    }
    if [message] =~ /lock/ {
        grok {
            match => [ "message","(?<word>lock)" ]
        }  
    }
    if [message] =~ /storage/ {
        grok {
            match => [ "message","(?<word>storage)" ]
        }  
    }
    if [message] =~ /authentication failure/ {
        grok {
            match => [ "message","(?<word>authentication failure)" ]
        }  
    }
    if [message] =~ /memory bottleneck/ {
        grok {
            match => [ "message","(?<word>memory bottleneck)" ]
        }  
    }
    if [message] =~ /runtime error/ {
        grok {
            match => [ "message","(?<word>runtime error)" ]
        }  
    }
    if [message] =~ /snapshot/ {
        grok {
            match => [ "message","(?<word>snapshot)" ]
        }  
    }
    
    
    mutate {
        add_field => {"[hour]" => "%{+HH}"}
	add_field => {"[weekday]" => "%{+EEE}"}
    }
    
    date{
        locale => "en"
        match => ["timestamp","MMM d HH:mm:ss","MMM dd HH:mm:ss","ISO8601"]
    }
    
    mutate { replace => [ "message", "%{log_message}" ]  }
    
    mutate {
        remove_field => [ "time" ,"month","monthday","timestamp","log_message","@version"]
    }
    
\}

output {

    if "dump" in [message]{
        email{
            subject => "Dump on %{host}"
            to => "bvervaele@oxya.com"
            body => "Host: %{host}\n\nTime: %{@timestamp}\n\nLine of the error: %{message}"
            address => "smtp.gmail.com"
            port => 587
            username => "logserviceoxya@gmail.com"
            password => "oxya1234"
            use\_tls => true
        }
    }

    if "semaphore" in [message]{
        email{
            subject => "Semaphore on %{host}" 
            to => "bvervaele@oxya.com"
            body => "Host: %{host}\n\nTime: %{@timestamp}\n\nLine of the error: %{message}"
            address => "smtp.gmail.com"
            port => 587
            username => "logserviceoxya@gmail.com"
            password => "oxya1234"
            use\_tls => true
        }
    }
    
    elasticsearch {
        hosts => ["localhost:9200"]
        user => elastic
        password => changeme
        index => "linux"
    }
    stdout { 
        codec => rubydebug
    }
}

\end{lstlisting}

Windows:
\lstset{escapechar=@,style=customc}        
\begin{lstlisting}[frame=single]
input {
  beats {
    host=>"localhost"
    port=>"2056"
  }
}
 
filter {
	mutate {
        remove_field => [ "Category" 
,"ComputerName","EventIdentifier","InsertionStrings","logfile","RecordNumber","SourceName","TimeGenerated","TimeWritten","EventType","RecordNumber","type"]
    }
}
output {
    
    elasticsearch {
         hosts => ["localhost:9200"]
        user => elastic
        password => changeme
        index => "windows"
    }
    stdout {
        codec => rubydebug
    }
}
\end{lstlisting}

Oracle:
\lstset{escapechar=@,style=customc}        
\begin{lstlisting}[frame=single]  
input {
    file {
        path => "/oracle/FDG/saptrace/diag/rdbms/fdg/FDG/trace/alert_FDG.log"
        codec => multiline {
            pattern => "%{DAY} %{MONTH} %{MONTHDAY} %{TIME} %{YEAR}"
            negate => true
            what => "previous"
	    auto_flush_interval => 1
        }
    }
}
 
filter {
 
    if [message] =~ /Starting ORACLE instance/ {
        mutate {
            add_field => [ "oradb_status", "starting" ]
        }
    } else if [message] =~ /Instance shutdown complete/ {
        mutate {
            add_field => [ "oradb_status", "shutdown" ]
        }
    } else {
        mutate {
            add_field => [ "oradb_status", "running" ]
        }
    }
 
    if [message] =~ /ORA-/ {
        grok {
            match => [ "message","(?<ORA->ORA-[0-9]*)" ]
        }  
    }

    if [message] =~ /current log/ {
        grok {
            match => [ "message","(?<word>current log)" ]
        }  
    } 
    if [message] =~ /deadlock/ {
        grok {
            match => [ "message","(?<word>deadlock)" ]
        }  
    }
    if [message] =~ /DATABASE OPEN/ {
        grok {
            match => [ "message","(?<word>DATABASE OPEN)" ]
        }  
    }  
    if [message] =~ /SYSTEM ARCHIVE LOG/ {
        grok {
            match => [ "message","(?<word>SYSTEM ARCHIVE LOG)" ]
        }  
    } 
    if [message] =~ /CONTROLFILE/ {
        grok {
            match => [ "message","(?<word>CONTROLFILE)" ]
        }  
    } 
    if [message] =~ /FULL/ {
        grok {
            match => [ "message","(?<word>FULL)" ]
        }  
    } 

   mutate {
        add_field => {"[hour]" => "%{+HH}"}
	add_field => {"[weekday]" => "%{+EEE}"}
    }
 

    grok {
        match => [ "message","%{DAY:day} %{MONTH:month} %{MONTHDAY:monthday} %{TIME:time} %{YEAR:year}(?<log\_message>.*$)" ]
    }
 
    mutate {
       add_field => {
            "timestamp" => "%{year} %{month} %{monthday} %{time}"
       }
    }

    date {
        timezone => "CET" 
        match => [ "timestamp" , "yyyy MMM dd HH:mm:ss" ]
    }
 
    mutate { replace => [ "message", "%{log_message}" ]  }
 
    mutate {
        remove_field => [ "time" ,"month","monthday","year","timestamp","day","log\_message"]
    }
}
 
output {
    if "ORA-00257" == [ORA-] or "ORA-16038" == [ORA-]{        
    email{
            subject => "Archiver Hung on %{host}"
            to => "bvervaele@oxya.com"
            body => "Host: %{host}\n\nTime: %{@timestamp}\n\nLine of the error: %{message}"
            address => "smtp.gmail.com"
            port => 587
            username => "logserviceoxya@gmail.com"
            password => "oxya1234"
            use_tls => true
        }
    }

    if "ORA-01114" == [ORA-] or "ORA-01157" == [ORA-] or "ORA-01578" == [ORA-] or "ORA-27048" == [ORA-]{
        email{
            subject => "Data Block Corruption" on %{host}
            to => "bvervaele@oxya.com"
            body => "Host: %{host}\n\nTime: %{@timestamp}\n\nLine of the error: %{message}"
            address => "smtp.gmail.com"
            port => 587
            username => "logserviceoxya@gmail.com"
            password => "oxya1234"
            use_tls => true
        }
    }

    if "ORA-01243" == [ORA-]{
        email{
            subject => "Media Failure on %{host}"
            to => "bvervaele@oxya.com"
            body => "Host: %{host}\n\nTime: %{@timestamp}\n\nLine of the error: %{message}"
            address => "smtp.gmail.com"
            port => 587
            username => "logserviceoxya@gmail.com"
            password => "oxya1234"
            use_tls => true
        }
    }

    if "ORA-01502" == [ORA-] or "ORA-20000" == [ORA-] {
        email{
            subject => "Invalid State on %{host}"
            to => "bvervaele@oxya.com"
            body => "Host: %{host}\n\nTime: %{@timestamp}\n\nLine of the error: %{message}"
            address => "smtp.gmail.com"
            port => 587
            username => "logserviceoxya@gmail.com"
            password => "oxya1234"
            use_tls => true
        }
    }

    if "ORA-01652" == [ORA-] or "ORA-01653" == [ORA-] or "ORA-01654" == [ORA-] {
        email{
            subject => "Extension Error on %{host}"
            to => "bvervaele@oxya.com"
            body => "Host: %{host}\n\nTime: %{@timestamp}\n\nLine of the error: %{message}"
            address => "smtp.gmail.com"
            port => 587
            username => "logserviceoxya@gmail.com"
            password => "oxya1234"
            use_tls => true
        }
    }

    if "ORA-00600" == [ORA-] or "ORA-07445" == [ORA-] or "ORA-04" in [ORA-]{
        email{
            subject => "Generic Error on %{host}"
            to => "bvervaele@oxya.com"
            body => "Host: %{host}\n\nTime: %{@timestamp}\n\nLine of the error: %{message}"
            address => "smtp.gmail.com"
            port => 587
            username => "logserviceoxya@gmail.com"
            password => "oxya1234"
            use_tls => true
        }
    }

    if "ORA-00020" == [ORA-]{
        email{
            subject => "Maximum number of processes exceeded on %{host}"
            to => "bvervaele@oxya.com"
            body => "Host: %{host}\n\nTime: %{@timestamp}\n\nLine of the error: %{message}"
            address => "smtp.gmail.com"
            port => 587
            username => "logserviceoxya@gmail.com"
            password => "oxya1234"
            use_tls => true
        }
    }
    elasticsearch {
        hosts => ["localhost:9200"]
        user => elastic
        password => changeme
        index => "oracle"
    }
    stdout { 
        codec => rubydebug
        id => "test outprint" 
    }
}
\end{lstlisting}

MSSQL:
\lstset{escapechar=@,style=customc}        
\begin{lstlisting}[frame=single]
input {
    file {
        path => "C:\Program Files\Microsoft SQL Server\MSSQL10_50.MSSQLSERVER\MSSQL\Log"
		codec => multiline {
            pattern => "%{YEAR:year}-%{MONTH:month}-%{MONTHDAY:monthday} %{TIME:time}"
            negate => true
            what => "previous"
			auto\_flush_interval => 1
        }
    }
}
 
filter {    
    grok {
        match => {
            "message" => "%{YEAR:year}-%{MONTH:month}-%{MONTHDAY:monthday} %{TIME:time} %{USERNAME:username}\s*(?<log\_message>.*$)"
        }
    }

	mutate {
       add_field => {
            "timestamp" => "%{year} %{month} %{monthday} %{time}"
       }
    }
	
    date{
        locale => "en"
        match => ["timestamp","yyyyy mm dd HH:mm:ss","yyyyy mm dd HH:mm:ss","ISO8601"]
    }
    
    mutate { replace => [ "message", "%{log_message}" ]  }
    
    mutate {
        remove\_field => [ "log\_message","@version"]
    }
    
 hosts => ["localhost:9200"]
        user => elastic
        password => changeme
        index => "oracle"\}

output {

    if "dump" in [message]{
        email{
            subject => "Dump"
            to => "bvervaele@oxya.com"
            body => "Host: %{host}\n\nTime: %{@timestamp}\n\nLine of the error: %{message}"
            address => "smtp.gmail.com"
            port => 587
            username => "logserviceoxya@gmail.com"
            password => "oxya1234"
            use\_tls => true
        }
    }

    if "semaphore" in [message]{
        email{
            subject => "Semaphore"
            to => "bvervaele@oxya.com"
            body => "Host: %{host}\n\nTime: %{@timestamp}\n\nLine of the error: %{message}"
            address => "smtp.gmail.com"
            port => 587
            username => "logserviceoxya@gmail.com"
            password => "oxya1234"
            use\_tls => true
        }
    }
    
    elasticsearch {
        hosts => ["localhost:9200"]
        user => elastic
        password => changeme
        index => "mssql"
    }
    stdout { 
        codec => rubydebug
    }
}

\end{lstlisting}

\listoffigures

\end{document}
