%%=============================================================================
%% Voorwoord
%%=============================================================================

\chapter*{Voorwoord}
\label{ch:voorwoord}
%% TODO:
%% Het voorwoord is het enige deel van de bachelorproef waar je vanuit je
%% eigen standpunt (``ik-vorm'') mag schrijven. Je kan hier bv. motiveren
%% waarom jij het onderwerp wil bespreken.
%% Vergeet ook niet te bedanken wie je geholpen/gesteund/... heeft
Deze is bachelorproef geschreven binnen het kader van het verplicht opleidingsonderdeel Bachelorproef voor het behalen van een diploma bachelor in de Toegepaste Informatica.

\textbf{Stageplaats}

Het onderwerp werd voorgesteld door mijn stagebedrijf. Daarom daarvoor zou ik graag mijn stageplaats oXya willen bedanken. Niet alleen voor het onderwerp maar ook de tijd die ze besteed hebben om mij te helpen in dit onderzoek. 
\begin{itemize}
	\item Dhr. Martin Van Den Abeele (co-promotor) - Bij hem kon ik altijd terecht met mijn technische vragen en hij bezorgde mij nuttige data. Dit zorgde voor een prettige samenwerking in een zeer leerrijk proces.
   
   
   \item Dhr. Lukas Becue - Hij ielp met het onderhouden van de elastic stack en het creëren van testdata.
\end{itemize}

\textbf{Familie}

Ik zou graag mijn ouders bedanken voor de steun, niet alleen nu maar doorheen heel mijn studies.
Ook voor hun bijdrage tot het verbeteren van deze bachelorproef, en verbeteren in de ruime zin van het woord.

\textbf{Docenten aan de HoGent}
	\begin{itemize}
	\item Dhr. Bert Van Vreeckem (promotor) - Hij volgde deze bachelorproef op en verzorgde nuttige feedback die leidde tot het eindresultaat.
   
   
   \item Dhr. Buysse Jens - Voor het in goede banen laten lopen van dit opleidingsonderdeel en de goede communicatie er rond.
\end{itemize}

