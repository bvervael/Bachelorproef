%%=============================================================================
%% Voorwoord
%%=============================================================================

\chapter*{Voorwoord}
\label{ch:voorwoord}
%% TODO:
%% Het voorwoord is het enige deel van de bachelorproef waar je vanuit je
%% eigen standpunt (``ik-vorm'') mag schrijven. Je kan hier bv. motiveren
%% waarom jij het onderwerp wil bespreken.
%% Vergeet ook niet te bedanken wie je geholpen/gesteund/... heeft
Deze is paper geschreven binnen het kader van het verplicht opleidingsonderdeel Bachelorproef voor het behalen van een diploma bachelor in de Toegepaste Informatica.

\textbf{Stage plaats}

Het onderwerp werd voorgesteld door mijn stagebedrijf dus daarvoor zou ik mijn stageplaats oXya willen bedanken. Niet alleen voor het onderwerp maar ook de tijd die ze besteed hebben om mij te helpen in dit onderzoek. 
\begin{itemize}
	\item Dhr. Martin Van Den Abeele (co-promotor) - Bij hem kon ik altijd terecht met mijn technische vragen en hij bezorgde mij nuttige data. Dit zorgde voor een prettige samenwerking in een zeer leerrijk proces.
   
   
   \item Dhr. Lukas Becue - Hielp met het onderhouden van de elastic stack en het creëren van testdata.
\end{itemize}

\textbf{Familie}

Ik zou graag mijn ouders bedanken voor de steun, niet alleen nu maar doorheen heel mijn studies.
Ook voor hun bijdrage tot het verbeteren van deze bachelorproef, en verbeteren in de ruime zin van het woord.

\textbf{Docenten aan de HoGent}
	\begin{itemize}
	\item Dhr. Bert Van Vreeckem (promotor) - Volgde deze bachelorproef op en verzorgde nuttige feedback die lijde tot het eind resultaat.
   
   
   \item Dhr. Buysse Jens - Voor het in goeie banen laten lopen van dit opleidings onderdeel en de goeie communicatie er rond.
\end{itemize}

