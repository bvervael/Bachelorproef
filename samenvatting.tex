%%=============================================================================
%% Samenvatting
%%=============================================================================

%% TODO: De "abstract" of samenvatting is een kernachtige (~ 1 blz. voor een
%% thesis) synthese van het document.
%%
%% Deze aspecten moeten zeker aan bod komen:
%% - Context: waarom is dit werk belangrijk?
%% - Nood: waarom moest dit onderzocht worden?
%% - Taak: wat heb je precies gedaan?
%% - Object: wat staat in dit document geschreven?
%% - Resultaat: wat was het resultaat?
%% - Conclusie: wat is/zijn de belangrijkste conclusie(s)?
%% - Perspectief: blijven er nog vragen open die in de toekomst nog kunnen
%%    onderzocht worden? Wat is een mogelijk vervolg voor jouw onderzoek?
%%
%% LET OP! Een samenvatting is GEEN voorwoord!

%%---------- Samenvatting -----------------------------------------------------
%%
%% De samenvatting in de hoofdtaal van het document

\chapter*{\IfLanguageName{dutch}{Samenvatting}{Abstract}}

De analyse van logfiles is een onderwerp dat recent aan belang gewonnen heeft en oXya,de opdrachtgever achter de bachelorproef,had dit dan ook graag in meer detail bekeken. Hier zijn enkele tools voor te vinden maar de Elastic stack trok hun aandacht. Dit omwille van de schaalbaarheid met grote hoeveelheden data, het open source is en het feit dat het alles lijkt te omvatten wat voor hen gewenst is. De opzet van deze bachelorproef is om te onderzoeken wat de mogelijkheden met elastic stack zijn omtrend proactief monitoring aan de hand van log files. 

Na een requiements analyse werden hun vereisten neergeschreven in het hoofdstuk casus.
Er werd onderzocht in welke mate Elastic stack aan hun vereisten voldeed door het ïmplementeren van een proof of concept. De proof of concept werd uitgewerkt voor twee besturingssystemen en twee databanken.

Dit werd uitgewerkt voor de drie core elementen en vier extra tools. In Logstash werd de input bekeken en hoe deze genormaliseerd kan worden. En er werd aandacht besteed aan het schrijven van een config file. Voor Elasticsearch wordt een beeld geschets van hoe de databank juist werkt hoe een zoekopdracht uit te voeren. Elasticsearch maakt het mogelijk om enkele tools erg efficiënt te laten werken. Deze tools bevatten de functionaliteiten die nodig waren voor deze casus. In de grafische interface Kibana kunnen dan beduidende grafieken gemaakt worden.
Om sommige vereisten aan te tonen waren ook theoritische berekeningen nodig.

Elastic stack voldeed aan elke voorwaarde al is verder onderzoek naar een goeie anomaly detectie methode wel aangeraden. Er kwamen reeds enkele mogelijkheden aanbod maar deze zijn nog voor verbetering vatbaar.
Elastic stack is een snel groeiend product en zal in de toekomst alleen maar meer mogelijkheden omvatten.